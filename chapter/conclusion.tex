Verbal aspect is a prominent linguistic phenomenon, but also highly complex.

Attempts to classify it are often either too coarse-grained ARE UNSATISFACTORY OR SMTH

In this thesis I have investigated the performance of current tools for aspect classification, using a classification schema taken from a larger meaning representation framework: UMR, how these tools deal with aspectual ambiguity, and what we can learn from this behaviour.

I exhibited some of the uses of this fine-tuned model for linguistics, such as Slavic prefix clustering with respect to aspect, and telicity detection in verbs of motion. Furthermore, I empirically validated the typological hypothesis that Slavic languages WHAT, using the language-level entropy


Multilingual model didnt work so well

AMBIGUITY DEPENDS ON THE SCHEMA USED!!

Come back to questions

Main take-aways?