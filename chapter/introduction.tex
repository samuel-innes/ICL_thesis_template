\section*{Motivation}
Despite being one of the most studied areas in linguistics, I COULDNT FIND ANY statistical approaches to aspect. 
\section*{Why aspect?}
One may justifiably beg the question why a “computational approach” to aspect (OR INDEED TO ANY PROBLEM IN THEORETICAL LINGUISTICS) is necessary or indeed useful: in an age of LLMs

The use of such a study comes down to the purpose of computational linguistics as an area of study. Computational linguistics has changed a lot since its conception in the mid 20th century, at some points being closer to linguistics and at others (arguably including right now) being closer to computer science. However the position of the field lying at the intersection between more well-established and well-defined areas of study has led to a fruitful exchange of ideas between the disciplines.\footnote{Just to name a few examples: formal languages, artificial neural networks and SOMETHING ELSE!!!!!!}

One reason is of course the contribution to the linguistic community: computational approaches to language have SPURRED ON LOTS OF PROGRESS (cf Chomsky!!!). A 

In the true nature of the interdisciplinarity of the field I wish to WORK AT BOTH AIMS IN PARALLEL and show how they complement each other.

Importance on engineering side:
\begin{itemize}
    \item{Zero-shot performance of ChatGPT comparable with BERT \citep{zhong2023chatgpt}}
    \item{I also experienced poor (?) performance with own experiments}
    \item{but fine-tuning lead to large improvements}
\end{itemize}

The fact that fine-tuning can lead the model to 

Therefore the purpose of this study is two-fold: firstly to further explore the phenomenology of aspect using methods from computational linguistics such as neural embeddings, and secondly to look at how current state-of-the-art approaches deal with this phenomenon and investigate how this could be improved.

\section*{Main contributions}
\begin{itemize}
    \item First in-depth study using computational approaches to study the phenomenology of aspect 
    \item First probing of neural models applied to the task
    \item SOMETHING ELSE
\end{itemize}
\section*{Structure of this thesis}