\section{Aspect classification}
Aspect has received comparatively very little attention from the computational linguistics community\citep{friedrich-etal-2023-kind}, especially from the Natural Language Processing (NLP) part of the field.\footnote{As opposed to those with more emphasis on the \emph{linguistics} part of computational linguistics.} However there have been some works in recent years looking at this area.

\subsection{Rules-based aspect class prediction}
\citet{siegel-mckeown-2000-learning} develop a rules-based aspect classification of verbs using co-occurrence information from a corpus. The aspect class they aim to predict is an inherent class of a verb (i.e. lexical aspect) and hence they do not take into account the sentence context, which often has an effect on the aspect of the verb considered. This is one of the issues which makes aforementioned distinction between lexical and grammatical aspect so difficult in practice.\footnote{See WHERE IS THIS for a more in-depth discusion on coercion and underspecification and their consequences.} The aspect classification scheme they used was that of \citet{moens-steedman-1988-temporal}, a 5-way class distinction building on \citet{vendler57}.

Interestingly they use their results draw the following linguistic conclusions: HERE.

\citet{annotAndAutoClassOfAspectCat} do something cool too.

\citet{chen-etal-2021-autoaspect} do something.

\subsection{(L)LMs and aspect}
\citet{metheniti-etal-2022-time} are bros and it WORKS.

\section{Available datasets}

\section{Formal representations of aspect}
\section{Related areas}
\subsection{Situation entities}
Situation entities classification is the task of identifying different types of situations, which exist at a clausal level. The task is comes more from the tradition of discourse analysis, since it is important for discourse representation theory (DRT) to know for example which new referents are introduced to a discourse and or also to analyse temporal relationships. Works usually use the original 8 types introduced by \citet{Smith_2003}: events, states, generalizing sentences, generic sentences, facts, propositions, questions and imperatives. While 