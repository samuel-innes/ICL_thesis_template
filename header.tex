\documentclass[fontsize=12pt, paper=a4, headinclude, twoside=false, parskip=half+, pagesize=auto, numbers=noenddot, plainheadsepline, open=right, toc=listof, toc=bibliography]{scrreprt}
% Abstand über Chapter Überschrift verringern
\renewcommand*{\chapterheadstartvskip}{\vspace*{-10pt}}
% PDF-Kompression
\pdfminorversion=5
\pdfobjcompresslevel=1
% Allgemeines
\usepackage[automark]{scrlayer-scrpage} % Kopf- und Fußzeilen
\usepackage{amsmath} % Mathesachen
\newtheorem{definition}{Definition}
\usepackage[T1]{fontenc} % Ligaturen, richtige Umlaute im PDF
\usepackage[utf8]{inputenc}% UTF8-Kodierung für Umlaute usw
\usepackage{hyphenat}
% Schriften
%\usepackage{mathpazo} % Palatino für Mathemodus
%\usepackage{tgpagella} % auch sehr schöne Schriften
\usepackage{lmodern}
%Courier 
% \usepackage{mathptmx}
% \usepackage[scaled=.90]{helvet}
% \usepackage{courier}
\usepackage{setspace} % Zeilenabstand
\usepackage{adjustbox}
\usepackage{hhline}
\usepackage{tikz}
%\usepackage{gb4e}
\usetikzlibrary{fit, positioning, shapes.geometric, decorations.pathreplacing,positioning, arrows.meta}

\onehalfspacing % 1,5 Zeilen
% Kein Seitenumbruch bei neuem Chapter
\usepackage{etoolbox}
\makeatletter
\patchcmd{\chapter}{\if@openright\cleardoublepage\else\clearpage\fi}{}{}{}
\makeatother
% Gliederungstiefe
\setcounter{tocdepth}{3}
\setcounter{secnumdepth}{3} 
% Schriften-Größen
\setkomafont{chapter}{\Huge\sffamily} % Überschrift der Ebene
\setkomafont{section}{\Large\sffamily}
\setkomafont{subsection}{\large\sffamily}
\setkomafont{subsubsection}{\large\sffamily}
\setkomafont{chapterentry}{\large\sffamily} % Überschrift der Ebene in Inhaltsverzeichnis
\setkomafont{descriptionlabel}{\bfseries\sffamily} % für description Umgebungen
\setkomafont{captionlabel}{\small\sffamily}
\setkomafont{caption}{\small\sffamily}
%\setkomafont{captionof}{\small\sffamily}
\setkomafont{footnote}{\sffamily}
% Sprache: Deutsch, Englisch als default
\usepackage[german, english]{babel} % Silbentrennung
\usepackage[babel]{csquotes} % Anführungszeichen mit enquote
% Tabellen
\usepackage{multirow} % Tabellen-Zellen über mehrere Zeilen
\usepackage{multicol} % mehre Spalten auf eine Seite
\usepackage{tabularx} % Für Tabellen mit vorgegeben Größen
\usepackage{longtable} % Tabellen über mehrere Seiten
\usepackage{array}
\usepackage{setspace}
\usepackage{threeparttable}
%  Bibliographie
%\usepackage{bibgerm} % Umlaute in BibTeX
%\usepackage{authordate1-4}
\usepackage{natbib}
\bibliographystyle{unsrtnat}
% Tabellen
\usepackage{multirow} % Tabellen-Zellen über mehrere Zeilen
\usepackage{multicol} % mehre Spalten auf eine Seite
\usepackage{booktabs} % Für Tabellen mit bottom, mid, toprule (schöner als hline)
\usepackage{tabularx} % Für Tabellen mit vorgegeben Größen
\usepackage{array}
\usepackage{float}
% Bilder
\usepackage{graphicx} % Bilder
\usepackage{color} % Farben
%\usepackage[usenames,dvipsnames]{xcolor}
\graphicspath{{images/}}
\DeclareGraphicsExtensions{.pdf,.png,.jpg} % bevorzuge pdf-Dateien
\usepackage{subfigure} % mehrere Abbildungen nebeneinander/übereinander
\newcommand{\subfigureautorefname}{\figurename} % um \autoref auch für subfigures benutzen
% Bildunterschrift
\setcapindent{0em} % kein Einrücken der Caption von Figures und Tabellen
\setcapwidth[c]{0.9\textwidth}
\setlength{\abovecaptionskip}{0.2cm} % Abstand der zwischen Bild- und Bildunterschrift
% Quellcode
\usepackage{listings} % für Formatierung in Quelltexten
\definecolor{grau}{gray}{0.45}
\lstset{
	extendedchars=true,
	basicstyle=\tiny\ttfamily,
	%basicstyle=\footnotesize\ttfamily,
	tabsize=2,
	keywordstyle=\textbf,
	commentstyle=\color{grau},
	stringstyle=\textit,
	numbers=left,
	numberstyle=\tiny,
	% für schönen Zeilenumbruch
	breakautoindent  = true,
	breakindent      = 2em,
	breaklines       = true,
	postbreak        = ,
	prebreak         = \raisebox{-.8ex}[0ex][0ex]{\Righttorque},
}
% linksbündige Fußnoten
\deffootnote{1.5em}{1em}{\makebox[1.5em][l]{\thefootnotemark}}
% PDF
\usepackage[english,pdfauthor={Julia Kreutzer},pdfauthor={Julia Kreutzer},pdftitle={--}, breaklinks=true,pdfpagelabels,plainpages=false]{hyperref}
\usepackage[final]{microtype} % mikrotypographische Optimierungen
\usepackage{url}
\usepackage{pdflscape} % einzelne Seiten drehen können
\usepackage[all]{hypcap} % Beim Klicken auf Links zum Bild und nicht zu Caption gehen

\usepackage{epstopdf} %eps to pdf 

\usepackage{chngcntr}
\usepackage{gb4e}
\noautomath
 
\counterwithout{figure}{chapter}
\counterwithout{table}{chapter}
\counterwithout{equation}{chapter}

\typearea{14} % typearea am Schluss berechnen lassen, damit die Einstellungen oben berücksichtigt werden
% für autoref von Gleichungen in itemize-Umgebungen
\makeatletter

%\newcommand{\saved@equation}{}
%\let\saved@equation\equation
%\def\equation{\@hyper@itemfalse\saved@equation}
%\makeatother 

% argmax als Befehl in Formeln
\DeclareMathOperator*{\argmax}{argmax}

% Eigene Befehle %%%%%%%%%%%%%%%%%%%%%%%%%%%%%%%%%%%%%%%%%%%%%%%%%5
% Matrix
\newcommand{\mat}[1]{
      {\textbf{#1}}
}
\newcommand{\todo}[1]{
      {\colorbox{red}{ TODO: #1 }}
}
\newcommand{\todotext}[1]{
      {\color{red} TODO: #1} \normalfont
}

\newcommand{\todoref}[1]{
	  {\colorbox{yellow}{ REFERENCE: #1 }}
}

\newcommand{\info}[1]{
      {\colorbox{green}{ (INFO: #1)}}
}
% Hinweis auf Programme in Datei
\newcommand{\datei}[1]{
      {\ttfamily{#1}}
}
\newcommand{\code}[1]{
      {\ttfamily{#1}}
}
% bild mit defnierter Breite einfügen
\newcommand{\bild}[4]{
  \begin{figure}[!hbt]
    \centering
      \vspace{1ex}
      \includegraphics[width=#2]{images/#1}
      \caption[#4]{\label{img.#1} #3}
    \vspace{1ex}
  \end{figure}
}
% bild auf eigener Seite
\newcommand{\bildp}[4]{
  \begin{figure}[p]
    \centering
      \vspace{1ex}
      \includegraphics[width=#2]{images/#1}
      \caption[#4]{\label{img.#1} #3}
    \vspace{1ex}
  \end{figure}
}
% bild mit eigener Breite
\newcommand{\bilda}[3]{
  \begin{figure}[!hbt]
    \centering
      \vspace{1ex}
      \includegraphics{images/#1}
      \caption[#3]{\label{img.#1} #2}
      \vspace{1ex}
  \end{figure}
}


% Bild todo
\newcommand{\bildt}[2]{
  \begin{figure}[!hbt]
    \begin{center}
      \vspace{2ex}
	      \includegraphics[width=6cm]{images/todobild}
      \caption{\label{#1} \todotext{#2}}
      \vspace{2ex}
    \end{center}
  \end{figure}
}

\newcommand{\ra}[1]{\renewcommand{\arraystretch}{#1}}

%\texttt{\newcommand\Chapter[2]{
%  \chapter[{\scshape#1}: {\itshape#2}]{\scshape#1\\\normalsize\itshape --#2}
%}